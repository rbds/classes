%=======================02-713 LaTeX template, following the 15-210 template==================
%
% You don't need to use LaTeX or this template, but you must turn your homework in as
% a typeset PDF somehow.
%
% How to use:
%    1. Update your information in section "A" below
%    2. Write your answers in section "B" below. Precede answers for all 
%       parts of a question with the command "\question{n}{desc}" where n is
%       the question number and "desc" is a short, one-line description of 
%       the problem. There is no need to restate the problem.
%    3. If a question has multiple parts, precede the answer to part x with the
%       command "\part{x}".
%    4. If a problem asks you to design an algorithm, use the commands
%       \algorithm, \correctness, \runtime to precede your discussion of the 
%       description of the algorithm, its correctness, and its running time, respectively.
%    5. You can include graphics by using the command \includegraphics{FILENAME}
%
\documentclass[11pt]{article}
\usepackage{amsmath,amssymb,amsthm}
\usepackage{graphicx}
%\usepackage[margin=1in]{geometry}
\usepackage{fancyhdr}
\usepackage[framed,numbered,autolinebreaks,useliterate]{mcode}
\setlength{\parindent}{0pt}
\setlength{\parskip}{5pt plus 1pt}
\setlength{\headheight}{16pt}
\newcommand\question[2]{\vspace{.25in}\textbf{#1: #2}\vspace{.5em}\hrule\vspace{.10in}}
\renewcommand\part[1]{\vspace{.10in}\textbf{(#1)}}
\pagestyle{fancyplain}
\lhead{\Large \textbf{Randy Schur}}
\chead{\LARGE \textbf{MAE 6254 HW1}}
\rhead{Due: 2/8/16}
\begin{document}\raggedright

\question{Problem 1}{} 

\part{a} 
\begin{equation*}
\frac{dy(\tau)}{d\tau} = \frac{1}{a}x'\left( \frac{\tau}{a} \right) = \frac{f\left( \frac{\tau}{a} \right)}{a}
\end{equation*}

\part{b}
\begin{equation*}
\frac{dy(\tau)}{d\tau}\Bigr|_{a=-1} = -f(-t)
\end{equation*}

\question{Problem 2}{}
\part{a}
\begin{align*}
x &= 
\begin{bmatrix}
\theta \\
\dot{\theta}
\end{bmatrix} \\
\dot{x} = f(x)  &= 
\begin{bmatrix}
x_2 \\
\frac{g}{l}sin(x_1) - cx_2
\end{bmatrix}
\end{align*}

\part{b} equilibria are at $\mathbf{\dot{x}} = 0$
\begin{align*}
\Rightarrow x_2 &= 0 \\
0 &= -\frac{g}{l}sin(x_1) - cx_2 \\
\Rightarrow x_1 &= 0, \pi
\end{align*}



\end{document}
