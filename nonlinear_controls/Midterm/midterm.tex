%%%%%%%%%%%%%%%%%%%%%%%%%%%%%%%%%%%%%%%%%
% Short Sectioned Assignment
% LaTeX Template
% Version 1.0 (5/5/12)
%
% This template has been downloaded from:
% http://www.LaTeXTemplates.com
%
% Original author:
% Frits Wenneker (http://www.howtotex.com)
%
% License:
% CC BY-NC-SA 3.0 (http://creativecommons.org/licenses/by-nc-sa/3.0/)
%
%%%%%%%%%%%%%%%%%%%%%%%%%%%%%%%%%%%%%%%%%

%----------------------------------------------------------------------------------------
%	PACKAGES AND OTHER DOCUMENT CONFIGURATIONS
%----------------------------------------------------------------------------------------

\documentclass[paper=a4, fontsize=11pt]{scrartcl} % A4 paper and 11pt font size

\usepackage[T1]{fontenc} % Use 8-bit encoding that has 256 glyphs
\usepackage[english]{babel} % English language/hyphenation
\usepackage{amsmath,amsfonts,amsthm} % Math packages
\usepackage{lipsum} % Used for inserting dummy 'Lorem ipsum' text into the template

\usepackage{sectsty} % Allows customizing section commands
\allsectionsfont{\centering \normalfont\scshape} % Make all sections centered, the default font and small caps

\usepackage{fancyhdr} % Custom headers and footers
\pagestyle{fancyplain} % Makes all pages in the document conform to the custom headers and footers
\fancyhead{} % No page header - if you want one, create it in the same way as the footers below
\fancyfoot[L]{} % Empty left footer
\fancyfoot[C]{} % Empty center footer
\fancyfoot[R]{\thepage} % Page numbering for right footer
\renewcommand{\headrulewidth}{0pt} % Remove header underlines
\renewcommand{\footrulewidth}{0pt} % Remove footer underlines
\setlength{\headheight}{13.6pt} % Customize the height of the header

\renewcommand{\thesubsection}{\thesection.\alph{subsection}}

\numberwithin{equation}{section} % Number equations within sections (i.e. 1.1, 1.2, 2.1, 2.2 instead of 1, 2, 3, 4)
\numberwithin{figure}{section} % Number figures within sections (i.e. 1.1, 1.2, 2.1, 2.2 instead of 1, 2, 3, 4)
\numberwithin{table}{section} % Number tables within sections (i.e. 1.1, 1.2, 2.1, 2.2 instead of 1, 2, 3, 4)

\setlength\parindent{0pt} % Removes all indentation from paragraphs - comment this line for an assignment with lots of text

%----------------------------------------------------------------------------------------
%	TITLE SECTION
%----------------------------------------------------------------------------------------

\newcommand{\horrule}[1]{\rule{\linewidth}{#1}} % Create horizontal rule command with 1 argument of height

\title{	
\normalfont \normalsize 
%\textsc{university, school or department name} \\ [25pt] % Your university, school and/or department name(s)
\horrule{0.5pt} \\[0.4cm] % Thin top horizontal rule
\huge MAE 6254 Midterm Exam \\ % The assignment title
\horrule{2pt} \\[0.5cm] % Thick bottom horizontal rule
}

\author{Randy Schur} % Your name

\date{\normalsize March 28 2016} % Today's date or a custom date

\begin{document}

\maketitle % Print the title

%----------------------------------------------------------------------------------------
%	PROBLEM 1
%----------------------------------------------------------------------------------------

\section{Problem 1}
For the following system: 
\begin{align*} 
\dot{x_1} &= -x_1^3 + x_2 \\
\dot{x_2} &= x_1 - x_2^3				
\end{align*}
a) find three equilibria \\
b) Find the type of each equilibrium
\subsection{}
Equilibria are at $x^*$ where $\dot{x}^*=0$. Therefore 
\begin{align*}
0 &= -x_1^3 + x_2 \\
0 &= x_1 - x_2^3
\end{align*}
This is true at:
\begin{align}
x^* = \begin{bmatrix}0 & 0\end{bmatrix}^T 
x^* = \begin{bmatrix}1 & 1\end{bmatrix}^T 
x^* = \begin{bmatrix}-1 & -1\end{bmatrix}^T 
\end{align}
\subsection{}
\begin{align*}
x &= x^* + \delta x \\
\dot{x} &= \dot{x}^* + \delta \dot{x} = \frac{\partial f}{\partial x}\Bigr|_{x^*} \\
A &= \begin{bmatrix}
\frac{\partial f_1}{\partial x_1} & \frac{\partial f_1}{\partial x_2} \\ 
\frac{\partial f_2}{\partial x_1} & \frac{\partial f_2}{\partial x_2} \\ 
\end{bmatrix}
= \begin{bmatrix}
-3x_1^2 & 1 \\ 
1 & -3x_2^2 
\end{bmatrix}
\end{align*}
By evaluating matrix A at each equilibrium and finding it's eigenvalues, we can determine the type of equilibrium.\\
Equilibrium 1: 
\begin{align}
A &= \begin{bmatrix}
0 & 1\\ 
1 & 0  
\end{bmatrix}\\
\lambda &= -1,\ 1 \Rightarrow\ saddle\ point
\end{align} \\
Equilibrium 2: 
\begin{align}
A &= \begin{bmatrix}
-3 & 1\\ 
1 & -3  
\end{bmatrix}\\
\lambda &= -4,\ -2 \Rightarrow\ stable\ node
\end{align} \\
Equilibrium 3: 
\begin{align}
A &= \begin{bmatrix}
-3 & 1\\ 
1 & -3  
\end{bmatrix}\\
\lambda &= -4,\ -2 \Rightarrow\ stable\ node
\end{align} \\

\newpage

\section{Problem 2}
a) Find the equilibrium of the system:
The equilibrium is at $x^* = \begin{bmatrix}0 & 0\end{bmatrix}^T$. This makes 
\begin{align*}
\dot{x_1} &=(1+0)(0-0) = 0\\
\dot{x_2} &= 0(1+0) = 0
\end{align*}

b) Make the strongest possible statement about the stability of the system using the given Lyapunov equation:
\begin{equation}
V(x_1, x_2) = \frac{x_1^2}{1+x_1^2}+\frac{x_2^2}{1+x_2^2}
\end{equation}
$V$ is positive definite becauase $V=0$ only if $x^= \begin{bmatrix}0 & 0\end{bmatrix}^T$.
\begin{align}
\dot{V} &= \frac{\partial V}{\partial x_1}\dot{x_1} + \frac{\partial V}{\partial x_2}\dot{x_2} \\
&= \frac{(1+x_1^2)2x_1 - x_1^2(2x_1}{(1+x_1^2)^2}(1+x_1^2)^2(-x_1-x_2) + \frac{(1+x_2^2)2x_2 - x_2^2(2x_2}{(1+x_2^2)^2}x_1(1+x_1^2)^2 \\
&= (2x_1+2x_1^3 - 2x_1^3)(-x_1-x_2) + (2x_2+2x_2^3-2x_2^3)x_1\\
&= -2x_1^2
\end{align}
Therefore $\dot{V}$ is negative semi-definite, and the equilibrium is stable. We can use LaSalle's theorem to show that the equilibrium of this time-invariant system is asymptotically stable.

Let $S=\{x \in D | x_1=0 \}$. Let $x_1, x_2$ be solutions staying in $S$. $V=\dot{V}=0$ implies that $x_1=0$, and therefore $\dot{x_1}=0$. This leaves the equation for $V$ as:
\begin{equation}
0 = \frac{x_2^2}{1+x_2^2}
\end{equation}
The only solution for which this is true is $x_2=0$. By LaSalle's theorem, the equilibrium is asymptotically stable. 

The above is true for $x \in D = \mathbb{R}^2$, and additionally V is radially unbounded. Therefore, the equilibrium is globally asymptotically stable.
\newpage

\section{Problem 3}
a) Show that the given Lyapunov equation is positive definite (p.d.).
\begin{align}
V(x_1, x_2) &= \frac{3}{2} x_1^2-x_1x_2+x_2^2 \\
&= \begin{bmatrix}x_1 & x_2\end{bmatrix} \mathbf{P} \begin{bmatrix}x_1 & x_2\end{bmatrix}^T \\
&= \begin{bmatrix}x_1 & x_2\end{bmatrix} \begin{bmatrix}3/2 & -2\\ 1 & 1\end{bmatrix} \begin{bmatrix}x_1 & x_2\end{bmatrix}^T
\end{align}
$V$ is p.d. if $\mathbf{P}$ is p.d. Matrix $\mathbf{P}$ is p.d. if the eigenvalues of $\mathbf{P}+\mathbf{P}^T/2>0$, or equivalently if the determinant of each leading principle minor is positive.
\begin{equation}
[P+P^T]/2 = Q = \begin{bmatrix}3/2 & -\frac{1}{2}\\ -\frac{1}{2} & 1\end{bmatrix}
\end{equation}
Both leading principle minors of $\mathbf{Q}$ are positive, and therefore $V$ is positive definite.
\newpage

\newpage
\section{Lists}

%------------------------------------------------

\subsection{Example of list (3*itemize)}
\begin{itemize}
	\item First item in a list 
		\begin{itemize}
		\item First item in a list 
			\begin{itemize}
			\item First item in a list 
			\item Second item in a list 
			\end{itemize}
		\item Second item in a list 
		\end{itemize}
	\item Second item in a list 
\end{itemize}

%------------------------------------------------

\subsection{Example of list (enumerate)}
\begin{enumerate}
\item First item in a list 
\item Second item in a list 
\item Third item in a list
\end{enumerate}

%----------------------------------------------------------------------------------------

\end{document}