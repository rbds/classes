\documentclass{article}
\usepackage{amsmath}
\usepackage{verbatim}
\usepackage{url}
\usepackage{amssymb}
\usepackage{bm}
\usepackage{times}
\usepackage{graphicx}

\setlength{\oddsidemargin}{-0.05in}
\setlength{\evensidemargin}{0.00in}
\setlength{\textwidth}{6.5in}
\setlength{\textheight}{9in}
\setlength{\topmargin}{-.5in}

\providecommand{\inv}[1]{{#1}^{\ensuremath{\mathsf{-1}}}} % inverse ...
\providecommand{\tr}[1]{{#1}^{\ensuremath{\mathsf{T}}}} % transpose ...
\providecommand{\abs}[1]{\lvert#1\rvert}
\providecommand{\norm}[1]{\left\lVert#1\right\rVert}
\providecommand{\vect}[1]{\bm#1}
\providecommand{\mat}[1]{\mathbf#1}
\providecommand{\bigTheta}{\mathit{\Theta}}
\providecommand{\bigO}{\mathit{O}}
\providecommand{\algorithmtopspace}

\providecommand{\ans}[1]{\vspace{.2in}\\\textbf{Ans:}#1\\}
%\providecommand{\ans}[1]{}

\title{CS 4/6545 Autonomous Robotics: \\Simulating the dynamics of a box undergoing compliant contact} 
\date{}

\begin{document}
\maketitle

\textbf{Preface: why this assignment?}
Contact is the key to manipulation, and the only way that Robotics currently
knows how to do any manipulation is by modeling that contact (to some degree
of fidelity). Contact is difficult to model quickly, accurately, and robustly.
Previous assignments have focused on control of robotic systems. \emph{This 
assignment will focus on modeling a ``robotic'' (rigid body) system.}\\

We will use \textsc{MATLAB} to simulate a $1m$ tall by $1m$ wide box 
contacting a ``ground plane''
(the line segment $y=0$) in 2D. While this scenario seems very simple, these
assumptions (planar ground, simple rigid body, 2D, compliant contact) are often 
used in Robotics.\\

Read all instructions carefully. Your submission will consist of all
code and plots. 

\begin{enumerate}
\item Implement the MATLAB function \texttt{drot(.)} (located in the file \texttt{drot.m}), which will return the $2 \times 2$ time derivative of a $2 \times 2$ rotation matrix; differentiate the matrix in \texttt{rot.m} with respect to time 
(don't forget that $\theta$ is a function of time!) This problem should give you a fairly gentle introduction to MATLAB syntax. You will need this function
in step \#3.

\textbf{Test your function:}
Do this by using two nearby values of $\theta$, which I will call $\theta_1$ and $\theta_2$. We can then define $\dot{\theta}$ (again, for testing purposes) as
$\theta_2 - \theta_1$. Does \texttt{drot}($\theta_1, \dot{\theta}$) $\approxeq R(\theta_2) - R(\theta_1)$?

\textbf{Explain how I arrived at the test above.} 

\item Fill out the necessary parts of \texttt{boxode.m}. Like \texttt{springode.m} that we developed in class, \texttt{boxode.m} evaluates an ordinary differential equation \emph{and} the 2nd order ODE is converted to a system of first order ODEs:
\begin{align}
\mat{M}\ddot{\vect{x}} = \vect{f}
\end{align} 
becomes
\begin{align}
\mat{M}\dot{\vect{v}} & = \vect{f} \\
\dot{\vect{x}} & = \vect{v}
\end{align} 
where $\mat{M}$ is the $3 \times 3$ \emph{generalized inertia} matrix, $\vect{x}$ is a three dimensional vector describing the position (two components) and orientation (one component) of the box in 2D, and $\vect{f}$ is the three dimensional vector of all forces on the box. \textbf{\emph{NOTE: I have included a file \texttt{boxode.m.hints} in the archive to help you along, if you want said help.}}

\item Test your simulation of the box by simulating the box. Call the \textsc{MATLAB} function \texttt{simulate(.)} as follows from the \textsc{MATLAB} prompt (make sure \textsc{MATLAB}'s working directory has your \texttt{.m} files in it):
\begin{verbatim}
>> [t, x] = simulate();
\end{verbatim}
Find reasonable gain constants that make the box sit on the ground. Plot the
height of the box using:
\begin{verbatim}
>> plot(t, x(2))
\end{verbatim}
\item Give the box some initial horizontal and vertical velocities and plot
the position of the box again (start velocity values near zero). Frictional forces should cause the box to
eventually come to a halt. Does it? (You may have to simulate for longer than
ten seconds to see this; you can change how long the simulation runs by
modifying \texttt{simulate.m}).
Use: 
\begin{verbatim}
>> plot(t, x(1));  
\end{verbatim}
to plot the horizontal position of the box and the command from above to plot the vertical position of the box. \emph{Hint:} you likely don't want a corner of the box to leave the ground plane (as it is likely to do under certain conditions). 
\end{enumerate}

\end{document}


